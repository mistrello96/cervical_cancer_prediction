\section{INTRODUCTION}
Il cancro alla cervice uterina è la terza forma di tumore più diffusa tra le donne, dopo quelle al seno e al colon-retto. La patologia aggredisce le cellule del collo dell’utero, ovvero il segmento che pone in collegamento l’utero con la vagina. Rispetto ad altre neoplasie, il tumore della cervice uterina ha il vantaggio di essere del tutto prevenibile e comunque ben curabile se diagnosticato precocemente \cite{veronesi}.
Nonostante la possibilità di prevenzione offerta da uno screening citologico regolare, questa forma di tumore colpisce oltre mezzo milione di donne all'anno, con un tasso di mortalità di oltre il $50\%$; la maggioranza dei decessi avviene in aree del mondo poco sviluppate, dove non è disponibile un meccanismo di screening regolare e sistematico sulla popolazione, che permetterebbe un riconoscimento precoce delle lesioni che precedono il tumore, permettendo così un intervento tempestivo dei medici \cite{paper}.
I maggiori fattori di rischio, che favoriscono la comparsa di questo tipo di cancro, sono individuabili principalmente nelle malattie sessualmente trasmissibili, in particolare nel Papilloma virus \cite{veronesi}.\\
In questo lavoro, è stato utilizzato un dataset contenete i dettagli di anamnesi ed esami clinici relativi a un campione di $858$ donne al fine di stabilire se sia possibile realizzare un modello di predizione efficace al fine di stabilire l'esito della biopsia del tessuto uterino a partire dai dati raccolti.
In particolare, sono stati messi a confronto diversi modelli di apprendimento automatico considerati \textit{explainable}, ovvero in grado di motivare il risultato della predizione fornita, in quanto in ambito medico modelli \textit{black box} risultano spesso non adatti a causa della limitata interpretabilità da parte dei medici.
Gli obiettivi di questo lavoro possono essere riassunti come segue:
\begin{itemize}
	\item esplorazione, pulizia e \textit{preprocessing} del dataset;
	\item confronto di differenti tecniche di feature reduction;
	\item addestramento di vari modelli per predire se i soggetti sono affetti dalla patologia;
	\item confronto dei vari modelli al fine di determinare quello con performance migliori.
\end{itemize}
Il report è stato suddiviso nelle seguenti sezioni: (II) introduzione al dataset e operazioni di esplorazione e pulizia dei dati; (III) presentazioni dei metodi utilizzati per la creazione dei modelli predittivi; (IV) presentazione ed analisi dei risultati ottenuti; (V) conclusioni e possibili sviluppi futuri.
Si segnala per questioni di compatibilità che la versione di Knime utilizzata per l'esecuzione del workflow relativo al lavoro svolto è la 4.1.0, con integrazione Python per l'esecuzione di snippet grafici e Weka per la creazione di alcuni modelli di predizione. 