\section{DATASET}
\subsection{Data exploration}
Il dataset fornito presenta $858$ sample, ognuno associato ad un totale di $36$ feature, etichetta inclusa. Un elenco di queste ultime è riportata in Tabella \ref{tab:attributes}, dove viene altresì riportato il DataType associato ad ognuno di essi. I dati provengono da uno screening effettuato su una popolazione di donne in un ospedale di Caracas, Venezuela \cite{ML}, delle quali sono stati raccolti e riportati una serie di dati personali e clinici. In particolare, nel dataset è possibile individuare un gruppo di feature relative all'attività sessuale delle donne, al numero di partner ed al tipo di contraccettivo utilizzato (ormonale vs intrauterino -IUD-). Viene poi riportata una lunga lista di malattie sessualmente trasmissibili (STDs), il numero totale di patologie sessuali contratte e l'intervallo di tempo trascorso dalla prima e dall'ultima diagnosi di queste malattie. Sono infine presenti informazioni circa precedenti diagnosi di cancro o Papilloma Virus (DX) e i risultati di alcuni esami clinici eseguiti sulle pazienti. L'etichetta, ovvero la variabile di output del modello, è l'esito della Biopsia, un esame eseguito su una porzione del tessuto sospetto per accertare la presenza o meno di cellule tumorali attive. 
\begin{table}
	\centering
	\caption{Tabella riportante le feature presenti nel dataset ed il relativo tipo.}
	\label{tab:attributes}
	\begin{tabular}{|c|c|}
		\toprule 
		Feature & DataType \\ 
		\midrule 
		Age & Interval (Int) \\ 
		Number of sexual partners & Interval (Int) \\ 
		First sexual intercourse & Interval (Int) \\ 
		Num of pregnancies & Interval (Int) \\
		Smokes & Nominal (Bool) \\
		Smokes (years) & Ratio (Double) \\ 
		Smokes (packs/year) & Ratio (Double) \\ 
		Hormonal Contraceptives & Nominal (Bool) \\ 
		Hormonal Contraceptives (years) & Ratio (Double) \\ 
		IUD & Nominal (Bool) \\ 
		IUD (years) & Ratio (Double) \\ 
		STDs & Nominal (Bool) \\ 
		STDs (number) & Interval (Int) \\ 
		STDs:condylomatosis & Nominal (Bool) \\ 
		STDs:cervical condylomatosis & Nominal (Bool) \\ 
		STDs:vaginal condylomatosis & Nominal (Bool) \\ 
		STDs:vulvo-perineal condylomatosis & Nominal (Bool) \\ 
		STDs:syphilis & Nominal (Bool) \\ 
		STDs:pelvic inflammatory disease & Nominal (Bool) \\ 
		STDs:genital herpes & Nominal (Bool) \\ 
		STDs:molluscum contagiosum & Nominal (Bool) \\ 
		STDs:AIDS & Nominal (Bool) \\ 
		STDs:HIV & Nominal (Bool) \\ 
		STDs:Hepatitis B & Nominal (Bool) \\ 
		STDs:HPV & Nominal (Bool) \\ 
		STDs: Number of diagnosis & Interval (Int) \\
		STDs: Time since first diagnosis & Interval (Int) \\ 
		STDs: Time since last diagnosis & Interval (Int) \\ 
		Dx:Cancer & Nominal (Bool) \\ 
		Dx:CIN & Nominal (Bool) \\ 
		Dx:HPV & Nominal (Bool) \\ 
		Dx & Nominal (Bool) \\ 
		Hinselmann & Nominal (Bool) \\ 
		Schiller & Nominal (Bool) \\
		Citology & Nominal (Bool) \\ 
		Biopsy & Nominal (Bool) \\ 
		\bottomrule 
	\end{tabular} 
\end{table}

Lo studio della distribuzione dei valori (presenza o assenza di cellule tumorali) assunti dall'etichetta ha evidenziato un forte sbilanciamento delle classi, come rappresentato in Figura \ref{fig:biopsydistribution}, con solo $55$ sample ($6,4\%$ del totale) appartenenti alla classe positiva.
\begin{figure}
	\centering
	\includegraphics[width=1\linewidth]{images/biopsy_distribution}
	\caption{Distribuzione delle etichette all'interno del dataset. Il problema risulta fortemente sbilanciato verso la classe negativa.}
	\label{fig:biopsydistribution}
\end{figure}
A seguito della fase di \textit{Data cleaning}, presentata nella sottosezione successiva, è stato eseguito uno studio della correlazione tra le varie feature del dataset (inclusa l'etichetta). La Figura \ref{fig:corrmatrix} riporta la \textit{heatmap} associata ai valori di correlazione; da essa si evince la presenza di una forte correlazione tra le feature relative alle malattie sessualmente trasmissibili e tra gli esami di laboratorio e l'etichetta. Più in generale, è presente una correlazione elevata tra quelle coppie di feature che rappresentano il medesimo dato, in un caso tramite valore booleano, nell'altro mediante un dato numerico. Questa constatazione risulta in linea con le aspettative e suggerisce la presenza di informazione ridondante, risolvibile mediante l'utilizzo di tecniche di riduzione della dimensionalità.
\begin{figure}
	\centering
	\includegraphics[width=1\linewidth]{images/corr_matrix}
	\caption{Matrice di correlazione delle feature del dataset. Legenda:
		0) Age,
		1) Number of sexual partners,
		2) First sexual intercourse,
		3) Num of pregnancies,
		4) Smokes,
		5) Smokes (years),
		6) Smokes (packs/year),
		7) Hormonal Contraceptives,
		8) Hormonal Contraceptives (years),
		9) IUD,
		10) IUD (years),
		11) STDs,
		12) STDs (number),
		13) STDs:condylomatosis,
		14) STDs:cervical condylomatosis,
		15) STDs:vaginal condylomatosis,
		16) STDs:vulvo-perineal condylomatosis,
		17) STDs:syphilis,
		18) STDs:pelvic inflammatory disease,
		19) STDs:genital herpes,
		20) STDs:molluscum contagiosum,
		21) STDs:AIDS,
		22) STDs:HIV,
		23) STDs:Hepatitis B,
		24) STDs:HPV,
		25) STDs: Number of diagnosis,
		26) Dx:Cancer,
		27) Dx:CIN,
		28) Dx:HPV,
		29) Dx,
		30) Hinselmann,
		31) Schiller,
		32) Citology,
		33) Biopsy.}
	\label{fig:corrmatrix}
\end{figure}

È stata quindi effettuata un'analisi più approfondita sui sample della classe positiva, ovvero che sono risultati positivi alla biopsia, i cui esiti sono riportati in Figura \ref{fig:dist}.
Grazie a informazioni derivanti da siti informativi e di divulgazione, è stato appreso come questa particolare forma di cancro sia spesso causata da una degenerazione del Papilloma virus, e che oltre al fumo, anche le malattie sessualmente trasmissibili costituiscono ulteriori fattori di rischio \cite{veronesi}.
Dalla Figura \ref{subfig:dist-age} si evince come il tumore abbia una forte incidenza nella fascia d'età $[18-33]$ anni; questo conferma le informazioni raccolte sui siti di divulgazione, dove si afferma che la fascia di età in cui questo tumore è maggiormente diffuso è tra i $20$ e i $30$ anni.
Analizzando la Figura \ref{subfig:dist-smoke} non si trova invece conferma del fumo come un fattore di rischio rilevante: la maggioranza dei sample positivi non fuma, o fa un utilizzo minimo di sigarette durante l'anno.
Per quanto riguarda la relazione tra malattie sessualmente trasmissibili e lo sviluppo del cancro alla cervice, in Figura \ref{subfig:dist-stds} è possibile rilevare come oltre un terzo delle pazienti con biopsia positiva avessero effettivamente contratto nel passato almeno una malattia sessualmente trasmissibile.
Al contempo però, la Figura \ref{subfig:dist-hpv} sembra riportare un dato contrastante con le informazioni relative alla patologia: solo due pazienti malate hanno riportato di essere risultate positive al papilloma virus.\\
Affinché sia possibile interpretare correttamente questi dati, è necessario considerare due differenti condizioni: in primo luogo, il campione di pazienti presenti nel dataset che hanno sviluppato il cancro risulta essere molto esiguo (soli 54 individui).
Questo campione non permette quindi di poter trarre conclusioni o generalizzare circa la malattia, in quanto la ridotta dimensione potrebbe mostrare comportamenti o caratteristiche differenti dalla popolazione complessiva.
In secondo luogo, tutti i dati relativi all'anamnesi del paziente sono stati raccolti sotto forma di sondaggio; le risposte non sono state quindi verificate e si affidano unicamente alla memoria/consapevolezza dei vari soggetti (\textit{i.e.}, un paziente potrebbe aver risposto di non avere malattie sessualmente trasmissibili a causa dell'assenza di una diagnosi di quest'ultima o per motivi personali).
Gli studi effettuati devono quindi essere interpretati unicamente come un'analisi relativa alla distribuzione del campione analizzato e non possono essere generalizzati per trarre conclusioni circa il comportamento della malattia su una popolazione più ampia.
\begin{figure}[ht!]
	\centering
	\begin{tabular}{c}
		\subfloat[\label{subfig:dist-age} KDE (\textit{Kernel Density Estimation}) delle istanze appartenenti alla classe positiva al variare dell'età.]{\includegraphics[width = 1\linewidth]{images/distribution_among_age}} \\
		\subfloat[\label{subfig:dist-smoke}KDE delle istanze appartenenti alla classe positiva al variare del numero di pacchetti di sigarette consumati all'anno.]{\includegraphics[width = 1\linewidth]{images/distribution_among_smoke}}
	\end{tabular}
	\begin{tabular}{cc}
		\subfloat[\label{subfig:dist-stds}Istogramma rappresentante la distribuzione dei soggetti affetti dalla patologia che hanno avuto almeno un malattia sessualmente trasmissibile.]{\includegraphics[width = .45\linewidth]{images/distribution_among_stds}} &
		\subfloat[\label{subfig:dist-hpv}Istogramma rappresentante la distribuzione dei soggetti affetti dalla patologia che hanno contratto il Papilloma Virus (HPV).]{\includegraphics[width = .45\linewidth]{images/distribution_among_hpv}}
	\end{tabular}
	\caption{Distribuzione dei soggetti affetti dalla patologia rispetto a varie feature.}
	\label{fig:dist}
\end{figure}

\subsection{Data cleaning and preprocessing}
È stata dapprima indagata la presenza di valori mancanti nel dataset, rilevando come le due feature relative alla data della prima e ultima diagnosi delle malattie sessualmente trasmissibili presentassero un numero elevatissimo di valori nulli.
Per questo motivo è stato deciso di rimuovere dal dataset questi dati, in quanto un tentativo di imputazione per questi valori si sarebbe dovuto basare su un numero di sample davvero esiguo. Non è invece stata rilevata la presenza di sample con un elevato numero di feature mancanti.
Il dataset è stato quindi diviso secondo l'etichetta ed è stata applicata una tecnica di imputazione per i \textit{missing value} dei due sottogruppi così generati: le feature intere sono state sostituite con la mediana dei valori, quelli booleani con la moda, mentre per i valori continui è stata utilizzato il valore medio arrotondato all'intero più vicino.
Quest'ultima scelta è stata presa a seguito dell'analisi qualitativa dei dati, che ha mostrato come la stragrande maggioranza delle feature Double contenesse valori pari a $0$.
Per questo motivo, la media di tali feature è risultata essere prossima al valore zero e, al fine di non introdurre un'alterazione nei dati presenti, è stata preferita una media arrotondata ad una più classica media, in modo da arrotondare tali medie a $0$.
La tecnica di imputazione utilizzata potrebbe generare alcune inconsistenze logiche all'interno di alcuni record (\textit{e.g.}, l'attributo \textit{STDs} falso ma un valore di \textit{STDs number} diverso da 0), anche se risulta poco probabile; a causa della forte correlazione logica tra le feature e alla probabile presenza di legami non triviali tra di esse, una gestione dei missing value che mantenesse la piena consistenza non è stata possibile. 
La creazione di una pipeline di imputazione che garantisca la consistenza dei dati resta un punto di potenziale interesse, che verrà discusso negli sviluppi futuri in calce a questo documento.
Dopo aver riunito il dataset, è stata effettuata una ricerca di outlier statistici, considerando un valore anomalo se a una feature è associato un valore che supera tre volte lo scarto interquantile della medesima feature.
Per fare ciò sono state considerate unicamente le feature relative all'età delle pazienti e ai dati sull'attività sessuale. Non sono stati inclusi i dati circa le abitudini relative al consumo di sigarette o agli anticoncezionali, in quanto si presenta un elevato numero di sample con valori associati a queste feature pari a $0$; il loro utilizzo avrebbe portato a considerare come outlier la maggioranza delle istanze con valori di questi campi non nulli.
A seguito della rimozione dei sample considerati outlier per almeno una feature, il dataset presenta un totale di 776 sample per la classe negativa e 54 sample per la classe positiva, evidenziando come la quasi totalità degli outlier sia stata individuata nella classe negativa.