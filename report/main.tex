\documentclass[a4paper, 12pt, conference]{ieeeconf} 
\overrideIEEEmargins
\usepackage[italian]{babel} % imposta lingua
\usepackage[utf8]{inputenc} % imposta lingua
\usepackage[babel]{csquotes} % imposta lingua
\usepackage[T1]{fontenc} % imposta lingua
\usepackage{amsmath, amssymb, amsfonts} % pachetto per formule
% \usepackage[parfill]{parskip} % non si dovrebbe fare, ma sostituisce le rientranze dei paragrafi con interlinea
\usepackage{listings} % per poter far riconoscere e colorare codice 
\usepackage{xcolor} % pacchetto per testo colorato
\usepackage{float, subfig} % pachetto per figure, per posizionamento
\usepackage{booktabs} % pacchetto per tabelle
\usepackage{graphicx, wrapfig} % pachetto per tabelle
\usepackage{tcolorbox} % riquadri colorati
\usepackage[Listato]{algorithm} % pseudocodice
\usepackage{algpseudocode} % pseudocodice
\usepackage[hidelinks]{hyperref} % indice e riferimenti cliccabili e senza riquadro rosso
\frenchspacing %spaziatura italiana per accenti
\usepackage[colorinlistoftodos,prependcaption,textsize=tiny]{todonotes} % note TODO
% CONFIGURAZIONE LINK E RIFERIMENTI
\hypersetup{%
	pdfpagemode={UseOutlines},
	bookmarksopen,
	pdfstartview={FitH},
	colorlinks,
	linkcolor={black}, %COLORE DEI RIFERIMENTI AL TESTO
	citecolor={blue}, %COLORE DEI RIFERIMENTI ALLE CITAZIONI
	urlcolor={blue} %COLORI DEGLI URL
}

\title{\LARGE \bf
Performance Time Report
}

\author{Mistri Matteo - 808097\\
	Daniele Maria Papetti - 808027
}


\begin{document}


\twocolumn[
\begin{@twocolumnfalse} 
\maketitle
\thispagestyle{empty}
\pagestyle{empty}
\rule{\textwidth}{.5pt}
\begin{abstract}

Il cancro alla cervice uterina è una delle forme di tumore più diffuse tra le donne, con un tasso di mortalità superiore al $50\%$.
Nonostante sia possibile individuarlo durante le fasi iniziali del suo sviluppo grazie a un continuo monitoraggio, in aree del mondo meno sviluppate ciò non è sempre possibile, portando così ad un incremento del tasso di mortalità.
%In questo lavoro si cerca di sviluppare un modello che ha lo scopo di aiutare i medici nel processo di diagnosi di tale patologia. 
L'obiettivo principale di questo lavoro è quello di produrre un modello di \textit{machine learning} in grado di predire se un soggetto, in funzione di un insieme di caratteristiche, sia affetto o meno dalla patologia.
Il dataset utilizzato contiene le informazioni di $858$ pazienti, i cui dati sono stati raccolti in un ospedale di Caracas \cite{ML}.
Studi esplorativi del dataset evidenziano come il problema sia estremamente sbilanciato (10 - 90) in favore della classe rappresentante l'assenza del tumore.
%Studi esplorativi del dataset evidenziano come il problema trattato sia binario (\textit{i.e.}, il paziente soffre o no della patologia) ed estremamente sbilanciato (10 - 90) in favore della classe rappresentante l'assenza del tumore.
Dopo aver effettuato alcune operazioni di \textit{preprocessing} sui dati, si è deciso di addestrare due differenti modelli (decision tree e random forest) per ognuno dei tre input generati.
%I modelli indagati sono stati i DT (decision tree) e le RF (random forest), in quanto risultano essere entrambi modelli \textit{explainable}; questa caratteristica risulta essere molto apprezzata in ambito medico, in quanto si è in grado di motivare le risposte fornite dal modello di apprendimento automatico.
I possibili input utilizzati per ogni modello sono il dataset senza nessuna operazione di feature reduction, il dataset dopo l'operazione di feature selection basata su correlazione e infine le feature estratte mediante il processo della Principal Component Analysis (PCA).
%Poiché sia la PCA che la feature selection sfruttano un iperparametro (\textit{e.g.}, il numero di componenti prodotti dalla PCA), sono stati effettuati degli studi preliminari su questi iperparametri in modo da selezionare dei relativi valori ottimali.
I vari modelli prodotti sono stati valutati secondo un procedimento di 5-fold \textit{stratified cross-validation}, utilizzando come metrica la f1-\textit{measure} rispetto alla classe minoritaria (\textit{i.e.}, biopsia positiva).
%Per favorire l'apprendimento del modello, i train set, ad ogni iterazione del processo, sono stati arricchiti mediante la tecnica di \textit{oversampling} SMOTE (Synthetic Minority Over-sampling TEchnique).
Il confronto tra i modelli è stato effettuato sfruttando dei \textit{paired} t-test, prima tra i due modelli appartenenti al medesimo input e poi tra i migliori di ogni strategia.
I risultati evidenziano l'assenza di differenze statistiche tra i modelli addestrati sfruttando l'intero dataset o con le feature prodotte dal processo di PCA, mentre le Random Forest (RF) risultano statisticamente migliori del Decision Tree (DT) nel caso dell'operazione di feature selection mediante correlazione; i confronti tra i modelli con input differenti non hanno evidenziato differenze statisticamente rilevanti.
%I confronti tra le RF ottenute dai vari metodi evidenziano che non è possibile stabilire differenze statisticamente significative tra esse.
Infine, gli algoritmi di apprendimento automatico utilizzati in precedenza sono stati addestrati sfruttando il dataset privato delle feature riguardanti gli esiti dei test clinici.
I risultati evidenziano che l'assenza di queste informazioni pregiudica fortemente la discriminazione della classe positiva da quella negativa, confermando l'importanza di tali esami.
\end{abstract}
\rule{\textwidth}{.5pt}
\end{@twocolumnfalse}
]

\section{INTRODUCTION}
Il cancro alla cervice uterina è la terza forma di tumore più diffusa tra le donne, dopo quelle al seno e al colon-retto. La patologia aggredisce le cellule del collo dell’utero, ovvero il segmento che pone in collegamento l’utero con la vagina. Rispetto ad altre neoplasie, il tumore della cervice uterina ha il vantaggio di essere del tutto prevenibile e comunque ben curabile se diagnosticato precocemente \cite{veronesi}.
Nonostante la possibilità di prevenzione offerta da uno screening citologico regolare, questa forma di tumore colpisce oltre mezzo milione di donne all'anno, con un tasso di mortalità di oltre il $50\%$; la maggioranza dei decessi avviene in aree del mondo poco sviluppate, dove non è disponibile un meccanismo di screening regolare e sistematico sulla popolazione, che permetterebbe un riconoscimento precoce delle lesioni che precedono il tumore, permettendo così un intervento tempestivo dei medici \cite{paper}.
I maggiori fattori di rischio, che favoriscono la comparsa di questo tipo di cancro, sono individuabili principalmente nelle malattie sessualmente trasmissibili, in particolare nel Papilloma virus \cite{veronesi}.\\
In questo lavoro, è stato utilizzato un dataset contenete i dettagli di anamnesi ed esami clinici relativi a un campione di $858$ donne al fine di stabilire se sia possibile realizzare un modello di predizione efficace al fine di stabilire l'esito della biopsia del tessuto uterino a partire dai dati raccolti.
In particolare, sono stati messi a confronto diversi modelli di apprendimento automatico considerati \textit{explainable}, ovvero in grado di motivare il risultato della predizione fornita, in quanto in ambito medico modelli \textit{black box} risultano spesso non adatti a causa della limitata interpretabilità da parte dei medici.
Gli obiettivi di questo lavoro possono essere riassunti come segue:
\begin{itemize}
	\item esplorazione, pulizia e \textit{preprocessing} del dataset;
	\item confronto di differenti tecniche di feature reduction;
	\item addestramento di vari modelli per predire se i soggetti sono affetti dalla patologia;
	\item confronto dei vari modelli al fine di determinare quello con performance migliori.
\end{itemize}
Il report è stato suddiviso nelle seguenti sezioni: (II) introduzione al dataset e operazioni di esplorazione e pulizia dei dati; (III) presentazioni dei metodi utilizzati per la creazione dei modelli predittivi; (IV) presentazione ed analisi dei risultati ottenuti; (V) conclusioni e possibili sviluppi futuri.
Si segnala per questioni di compatibilità che la versione di Knime utilizzata per l'esecuzione del workflow relativo al lavoro svolto è la 4.1.0, con integrazione Python per l'esecuzione di snippet grafici e Weka per la creazione di alcuni modelli di predizione. 

\section{DATASET}
\subsection{Data exploration}
Il dataset fornito presenta $858$ sample, ognuno associato ad un totale di $36$ feature, etichetta inclusa. Un elenco di queste ultime è riportata in Tabella \ref{tab:attributes}, dove viene altresì riportato il DataType associato ad ognuno di essi. I dati provengono da uno screening effettuato su una popolazione di donne in un ospedale di Caracas, Venezuela \cite{ML}, delle quali sono stati raccolti e riportati una serie di dati personali, medici e strumentali. In particolare, nel dataset è possibile individuare un gruppo di feature relative all'attività sessuale delle donne, al numero di partner ed al tipo di contraccettivo utilizzato (ormonale vs intrauterino -IUD-). Viene poi riportata una lunga lista di malattie sessualmente trasmissibili (STDs), il numero totale di patologie sessuali contratte e l'intervallo di tempo trascorso dalla prima e dall'ultima diagnosi di queste malattie. Sono infine presenti informazioni circa precedenti diagnosi di cancro o Papilloma Virus (DX) e i risultati di alcuni esami strumentali eseguiti sulle pazienti. L'etichetta, ovvero la feature che deve venir predetto dal modello, è l'esito della Biopsia, un esame eseguito su una porzione del tessuto sospetto per accertare la presenza o meno di cellule tumorali attive. 
\begin{table}
	\centering
	\caption{Tabella riportante le feature presenti nel dataset ed il relativo tipo.}
	\label{tab:attributes}
	\begin{tabular}{|c|c|}
		\toprule 
		Feature & DataType \\ 
		\midrule 
		Age & Interval (Int) \\ 
		Number of sexual partners & Interval (Int) \\ 
		First sexual intercourse & Interval (Int) \\ 
		Num of pregnancies & Interval (Int) \\
		Smokes & Nominal (Bool) \\
		Smokes (years) & Ratio (Double) \\ 
		Smokes (packs/year) & Ratio (Double) \\ 
		Hormonal Contraceptives & Nominal (Bool) \\ 
		Hormonal Contraceptives (years) & Ratio (Double) \\ 
		IUD & Nominal (Bool) \\ 
		IUD (years) & Ratio (Double) \\ 
		STDs & Nominal (Bool) \\ 
		STDs (number) & Interval (Int) \\ 
		STDs:condylomatosis & Nominal (Bool) \\ 
		STDs:cervical condylomatosis & Nominal (Bool) \\ 
		STDs:vaginal condylomatosis & Nominal (Bool) \\ 
		STDs:vulvo-perineal condylomatosis & Nominal (Bool) \\ 
		STDs:syphilis & Nominal (Bool) \\ 
		STDs:pelvic inflammatory disease & Nominal (Bool) \\ 
		STDs:genital herpes & Nominal (Bool) \\ 
		STDs:molluscum contagiosum & Nominal (Bool) \\ 
		STDs:AIDS & Nominal (Bool) \\ 
		STDs:HIV & Nominal (Bool) \\ 
		STDs:Hepatitis B & Nominal (Bool) \\ 
		STDs:HPV & Nominal (Bool) \\ 
		STDs: Number of diagnosis & Interval (Int) \\
		STDs: Time since first diagnosis & Interval (Int) \\ 
		STDs: Time since last diagnosis & Interval (Int) \\ 
		Dx:Cancer & Nominal (Bool) \\ 
		Dx:CIN & Nominal (Bool) \\ 
		Dx:HPV & Nominal (Bool) \\ 
		Dx & Nominal (Bool) \\ 
		Hinselmann & Nominal (Bool) \\ 
		Schiller & Nominal (Bool) \\
		Citology & Nominal (Bool) \\ 
		Biopsy & Nominal (Bool) \\ 
		\bottomrule 
	\end{tabular} 
\end{table}

Lo studio della distribuzione dei valori (presenza o assenza di cellule tumorali) assunti dall'etichetta ha riportato un forte sbilanciamento delle classi, come rappresentato in Figura \ref{fig:biopsydistribution}, con solo $55$ sample ($6,4\%$ del totale) appartenenti alla classe positiva.
\begin{figure}
	\centering
	\includegraphics[width=1\linewidth]{images/biopsy_distribution}
	\caption{Distribuzione delle etichette all'interno del dataset. Il problema risulta fortemente sbilanciato verso la classe negativa.}
	\label{fig:biopsydistribution}
\end{figure}
A seguito della fase di \textit{Data cleaning}, presentata nella sottosezione successiva, è stato eseguito uno studio della correlazione tra le varie feature del dataset (inclusa l'etichetta). La Figura \ref{fig:corrmatrix} riporta la \textit{heatmap} associata ai valori di correlazione; da essa si evince la presenza di una forte correlazione interna tra le feature relative alle malattie sessualmente trasmissibili e tra gli esami di laboratorio e l'etichetta. Più in generale, è presente una correlazione elevata tra quelle coppie di feature che rappresentano il medesimo dato, in un caso tramite valore booleano, nell'altro mediante un dato numerico. Questa costatazione risulta in linea con le aspettative e suggerisce una presenza di ridondanza dell'informazione, risolvibile mediante l'utilizzo di tecniche di riduzione della dimensionalità.
\begin{figure}
	\centering
	\includegraphics[width=1\linewidth]{images/corr_matrix}
	\caption{Matrice di correlazione delle feature del dataset. Legenda:
		0) Age,
		1) Number of sexual partners,
		2) First sexual intercourse,
		3) Num of pregnancies,
		4) Smokes,
		5) Smokes (years),
		6) Smokes (packs/year),
		7) Hormonal Contraceptives,
		8) Hormonal Contraceptives (years),
		9) IUD,
		10) IUD (years),
		11) STDs,
		12) STDs (number),
		13) STDs:condylomatosis,
		14) STDs:cervical condylomatosis,
		15) STDs:vaginal condylomatosis,
		16) STDs:vulvo-perineal condylomatosis,
		17) STDs:syphilis,
		18) STDs:pelvic inflammatory disease,
		19) STDs:genital herpes,
		20) STDs:molluscum contagiosum,
		21) STDs:AIDS,
		22) STDs:HIV,
		23) STDs:Hepatitis B,
		24) STDs:HPV,
		25) STDs: Number of diagnosis,
		26) Dx:Cancer,
		27) Dx:CIN,
		28) Dx:HPV,
		29) Dx,
		30) Hinselmann,
		31) Schiller,
		32) Citology,
		33) Biopsy.}
	\label{fig:corrmatrix}
\end{figure}

È stata quindi effettuata un'analisi più approfondita sui sample della classe positiva, ovvero che hanno ricevuto una diagnosi di cancro alla cervice uterina.
Grazie a informazioni apprese da siti informativi e di divulgazione \cite{veronesi}, è stato appreso come questa particolare forma di cancro sia spesso causata da una degenerazione del Papilloma virus, e che oltre al fumo, anche le malattie sessualmente trasmissibili costituiscono ulteriori fattori di rischio.
In Figura \ref{fig:dist} sono riportati i risultati delle analisi svolte sui sample con biopsia positiva.
Dalla Figura \ref{subfig:dist-age} si evince come il tumore abbia una forte incidenza nella fascia d'età $[18-33]$ anni; questo conferma le informazioni raccolte sui siti di divulgazione, dove si afferma che la fascia di età in cui questo tumore è maggiormente diffuso è tra i $20$ e i $30$ anni.
Analizzando la Figura \ref{subfig:dist-smoke} non troviamo invece conferma del fumo come un fattore di rischio rilevante: la maggioranza dei sample positivi non fuma, o fa un consumo minimo di sigarette durante l'anno.
Per quanto riguarda la relazione tra malattie sessualmente trasmissibili e lo sviluppo del cancro alla cervice, in Figura \ref{subfig:dist-stds} è possibile rilevare come oltre un terzo delle pazienti con biopsia positiva avessero effettivamente contratto nel passato almeno una malattia sessualmente trasmissibile.
Al contempo però, la Figura \ref{subfig:dist-hpv} sembra riportare un dato contrastante con le informazioni relative alla patologia: solo due pazienti malate hanno riportato di essere risultate positive al papilloma virus.\\
Al fine di interpretare correttamente questi dati, è necessario considerare due differenti condizioni: in primo luogo, il campione di pazienti presenti nel dataset che hanno sviluppato il cancro risulta essere molto esiguo (soli 54 individui).
Questo campione non permette quindi di poter trarre conclusioni o generalizzare circa la malattia, in quanto la ridotta dimensione potrebbe mostrare comportamenti o caratteristiche differenti dalla popolazione complessiva.
In secondo luogo, tutti i dati relativi all'anamnesi del paziente sono stati raccolti sotto forma di sondaggio; le risposte non possono essere quindi verificate e si affidano unicamente alla memoria/consapevolezza dei vari soggetti (\textit{i.e.}, un paziente potrebbe aver risposto di non avere malattie sessualmente trasmissibili a causa dell'assenza di una diagnosi di quest'ultima).
Gli studi effettuati devono quindi essere interpretati unicamente come un'analisi relativa alla distribuzione del campione analizzato e non possono essere generalizzati per trarre conclusioni circa il comportamento della malattia su una popolazione più ampia.
\begin{figure}[ht!]
	\centering
	\begin{tabular}{c}
		\subfloat[\label{subfig:dist-age} KDE (\textit{Kernel Density Estimation}) delle istanze appartenenti alla classe positiva al variare dell'età.]{\includegraphics[width = 1\linewidth]{images/distribution_among_age}} \\
		\subfloat[\label{subfig:dist-smoke}KDE delle istanze appartenenti alla classe positiva al variare del numero di pacchetti di sigarette consumati all'anno.]{\includegraphics[width = 1\linewidth]{images/distribution_among_smoke}}
	\end{tabular}
	\begin{tabular}{cc}
		\subfloat[\label{subfig:dist-stds}Istogramma rappresentante la distribuzione dei soggetti affetti dalla patologia che hanno avuto almeno un malattia sessualmente trasmissibile.]{\includegraphics[width = .45\linewidth]{images/distribution_among_stds}} &
		\subfloat[\label{subfig:dist-hpv}Istogramma rappresentante la distribuzione dei soggetti affetti dalla patologia che hanno contratto il Papilloma Virus (HPV).]{\includegraphics[width = .45\linewidth]{images/distribution_among_hpv}}
	\end{tabular}
	\caption{Distribuzione dei soggetti affetti dalla patologia rispetto a varie feature.}
	\label{fig:dist}
\end{figure}

\subsection{Data cleaning and preprocessing}
È stata dapprima indagata la presenza di valori mancanti nel dataset, rilevando come le due feature relative alla data della prima e ultima diagnosi di malattie sessualmente trasmissibili presentassero un numero elevatissimo di valori nulli.
Per questo motivo è stato deciso di rimuovere dal dataset questi dati, in quanto un tentativo di imputazione di questi valori si sarebbe dovuto basare su un numero di sample davvero esiguo. Non è invece stata rilevata la presenza di sample con un elevato numero di feature mancanti.
Il dataset è stato quindi diviso secondo l'etichetta ed è stata applicata una tecnica di imputazione per i \textit{missing value} dei due sottogruppi così generati: le feature intere sono state sostituite con la mediana dei valori, quelli booleani con la moda, mentre per i valori continui è stata utilizzato il valore medio arrotondato all'intero più vicino. Quest'ultima scelta è stata presa a seguito dell'analisi qualitativa dei dati, che ha mostrato come la stragrande maggioranza dei valori Double \todo[inline]{Sono confuso, se sono double vuol dire che hanno valori in R, quindi posso fare la media senza arrotondare, perché allora arrotondiamo? DP} risultasse essere pari a 0 o ad un valore intero. Per questo motivo, al fine di non introdurre un'alterazione nel tipo dei dati presenti, è stata preferita una media arrotondata ad una più classica media semplice.
Dopo aver riunito il dataset, è stata effettuata una ricerca di outlier statistici.
Per fare ciò sono stati considerati unicamente le feature relative all'età delle pazienti e ai dati sull'attività sessuale. Non sono stati inclusi i dati circa le abitudini relative al consumo di sigarette o relativi agli anticoncezionali, in quanto si presenta un elevato numero di sample con valori associati a queste feature pari a $0$; il loro utilizzo avrebbe portato a considerare la maggioranza delle istanze con valori non nulli in questi campi degli outlier. Sono stati considerati outlier di una feature quei valori che superano di tre volte lo scarto interquantile. 
A seguito della rimozione dei sample considerati outlier per almeno una feature, il dataset presenta un totale di 776 sample per la classe negativa e 54 sample per la classe positiva, evidenziando come la maggioranza degli outlier sia stata individuata nella classe negativa.

\section{METODI}
A seguito delle operazioni di \textit{preprocessing} del dataset, al fine di confrontare tra loro le varie tecniche di feature reduction, il \textit{workflow} si sviluppa in contemporanea su tre differenti livelli.
Ciascuno di essi gestisce privatamente una copia del dataset preprocessato.
Il primo livello fornisce in input agli algoritmi di apprendimento del modello il dataset senza che abbia subito alcuna operazione di feature reduction.
Il secondo fornisce come input all'algoritmo il risultato di un'operazione di feature extraction applicata al dataset mediante la tecnica della PCA. Questo approccio permette di trasformare un insieme di variabili correlate tra loro in un insieme ordinato di nuove feature non correlate tra loro; la trasformazione è definita in modo che la prima componente risulti essere quella con maggiore varianza, mentre tutte le successive mostrano il maggior valore di varianza possibile sotto il vincolo di ortogonalità con la precedente.
Al fine di stabilire la dimensionalità dello spazio di feature prodotto dalla PCA, è stato effettuato uno studio preliminare che consiste nella valutazione delle performance in 5-fold \textit{stratified cross validation} dei due algoritmi utilizzati al variare del numero di componenti nell'intervallo $[1, 33]$.
Infine, il terzo livello del \textit{workflow} fornisce come input agli algoritmi di apprendimento automatico il dataset a seguito di un'operazione di feature selection mediante l'utilizzo di un filtro basato sulla correlazione; al fine di individuare il threshold di selezione ottimale, è stato prodotto uno studio simile a quello realizzato per la PCA, testando valori i valori nel range [0,05, 0,95] e valutando le performance dei modelli.
La scelta di tale tecnica è stata dettata dal fatto che esso risulta appartenere alla classe dei filtri multivariati; ciò permette di individuare sia le feature irrilevanti sia quelle dipendenti tra loro, permettendo così di rimuovere la ridondanza dell'informazione. 
In aggiunta, i risultati evidenziati durante l'esplorazione del dataset (la presenza di feature altamente correlate tra loro) hanno suggerito la scelta di questa strategia, che risulta essere computazionalmente sostenibile dato il numero contenuto di sample presenti nel dataset.
Per ogni livello del \textit{workflow} sono stati utilizzati due differenti algoritmi di \textit{machine learning} per apprendere i rispettivi modelli; in particolare, sono stati impiegati alberi di decisione (a cui d'ora in poi ci si riferirà con DT in assenza di ambiguità) implementati da \textit{Weka} e random forest implementata nativamente in \texttt{Knime}.
Gli alberi di decisione sono una classe di tecniche di apprendimento automatico la cui struttura è composta da nodi interni che rappresentano test su una o più feature, mentre le foglie costituiscono i risultati della decisione.
L'algoritmo utilizzato per la creazione di questi alberi è \texttt{J48}, un'implementazione che estende la più classica \texttt{ID3} con il supporto delle feature con valori continui.
Al fine di stabilire quali siano i migliori test da effettuare all'interno dei nodi dell'albero viene valutata l'\textit{information gain}, una metrica in grado di evidenziare le divisioni in grado di generare sottogruppi nei quali l'entropia rispetto alla classe di appartenenza dei sample appartenenti al sottogruppo generato diminuisce rispetto all'entropia dell'insieme che li ha generati.
Le random forest sono un modello \textit{ensamble} costituito da un insieme predefinito di alberi di decisione, ognuno addestrato su un sottoinsieme differente di sample e feature.
Al momento della presentazione di un sample da classificare, ogni albero che compone la foresta propone la sua classificazione, che è poi valutata in modo pesato insieme a tutte le altre al fine di fornire una predizione complessiva.\\
Per confrontare le performance medie dei modelli derivanti dagli algoritmi appena presentati è stato sfruttato un processo di 5-fold \textit{stratified cross-validation}.
La scelta di utilizzare una tecnica di \textit{cross-validation} stratificata (SCV) è stata dettata dalla natura estremamente sbilanciata del problema, che ha portato altresì alla scelta di utilizzare un numero di fold contenuto, pari a $5$, in modo da mantenere in ogni fold un numero ragionevole di sample della classe minoritaria.
Data la scarsa rappresentazione della classe positiva presente nei vari train set, alla SCV è stata affiancata un'operazione di \textit{over-sampling} della classe minoritaria mediante la tecnica SMOTE. 
Questa tecnica permette di generare dei sample sintetici della classe minoritaria mediante la seguente procedura: per ogni campione della classe di minoranza vengono selezionati i \textit{k-nearest neighbors} del punto e si crea un sample tra il punto corrente e uno dei vicini scelto a caso. 
I valori delle feature del sample sintetico sono posizionate casualmente nello spazio che intercorre lungo la dimensione tra le due componenti dei due punti considerati; per far ciò si sfrutta l'estrazione di un numero casuale da una distribuzione uniforme nell'intervallo continuo $[0, 1]$.
L'operazione è stata effettuata esclusivamente sulla porzione di train di ogni iterazione della SCV, in modo da garantire che le performance misurate sul test set non fossero in alcuno modo influenzate.
L'utilizzo di una tecnica come SMOTE ha consentito di bilanciare le classi nei dati di training, permettendo così al modello di apprendere in modo più soddisfacente; è stato preferito l'\textit{over-sampling} all'\textit{under-sampling} in quanto quest'ultimo avrebbe ridotto drasticamente le dimensioni del train set, minando così le possibilità di apprendimento.
Si fa notare che il \textit{seed} per la generazione casuale dei fold è stato fissato in favore della ripetibilità dell'esperimento; tale \textit{seed} è inoltre condiviso tra tutti i generatori, in modo che vengano utilizzati i medesimi fold nei vari livelli del \textit{workflow}.
Al fine di stabilire se l'utilizzo da tale tecnica producesse un'alterazione delle performance nei modelli considerati, una verifica qualitativa è stata eseguita sui risultati degli algoritmi addestrati sull'intero dataset in presenza ed in assenza della tecnica di \textit{oversampling}.
Al termine delle operazioni di \textit{cross-validation}, i modelli prodotti dagli algoritmi sono stati confrontati tra loro basandosi sulla f1-\textit{measure} della classe minoritaria (\textit{i.e.}, presenza di cancro), calcolata come: \[2 \cdot \frac{precision \cdot recall}{precision+recall}.\]
L'utilizzo di una metrica come l'accuratezza o l'errore non s'addice al problema affrontato in quanto, considerato l'elevato grado di sbilanciamento del problema, un classificatore \textit{baseline} che predica esclusivamente l'assenza del tumore avrebbe ottenuto un'accuratezza molto elevata, pur non servendo allo scopo prefissato. 
Si è invece scelto di utilizzare una misura di performance come l'f1-\textit{measure} poiché in un problema di questo tipo è stato ritenuto rilevante considerare sia la \textit{precision}, ovvero la percentuale di veri positivi individuati rispetto al totale di predizioni positive fatte, sia la \textit{recall}, ovvero il numero di veri positivi individuati rispetto al numero totale che andava individuato. 
La motivazione alla base di questa decisione risulta essere triviale: si vorrebbero individuare correttamente il maggior numero possibile di soggetti affetti dalla patologia, mentre, al contempo, si vorrebbe evitare di diagnosticare erroneamente a molti pazienti di essere malati di cancro e costringerli così a dover affrontare ulteriori esami e operazioni invasive e costose; si è quindi optato per la medie armonica delle due metriche sopracitate (\textit{i.e.}, f1-\textit{measure}), in modo da bilanciare i falsi positivi con i falsi negativi.
Un incontro con esperti del dominio medico potrebbe portare ad una modifica della misura di performance utilizzata, prediligendo una delle due componenti dell'f1-\textit{measure} a discapito dell'altra; l'introduzione di una matrice dei costi che stimi il differente impatto economico e umano di falsi positivi e negativi potrebbe rivelarsi particolarmente utile in questo senso.\\
I confronti fra i vari modelli sono stati effettuati valutando le differenze tra la medie degli \textit{score} prodotti durante le varie iterazioni della \textit{cross-validation}; per stabilire se tali differenze fossero statisticamente significativa o meno sono stati utilizzati dei \textit{paired} t-test con un livello di confidenza pari a 0.95.
È stata utilizzata la versione \textit{paired} del test in quanto le performance dei vari modelli derivano i medesimi fold del processo di SCV.
Successivamente, sono stati confrontati fra loro i modelli ritenuti migliori per ogni livello del workflow (se sono risultati statisticamente differenti, altrimenti ne è stato selezionato uno a piacere), andando a verificare la presenza di un modello che risulti statisticamente migliore rispetto agli altri; anche questi test sono stati effettuati mediante un \textit{paired} t-test con un valore di confidenza fissato a 0.95.

Infine, sono state valutate le performance dei due algoritmi di apprendimento automatico proposti in assenza di alcune feature derivanti da esami clinici invasivi e costosi che, soprattutto in paesi di via di sviluppo, non risultano sempre disponibili o di facile accesso.
In particolare, dal dataset ottenuto a seguito della fase di \textit{preprocessing} sono state rimosse le feature relative ai 3 esami clinici effettuati: \textit{Hinselmann} (cioè la colposcopia), \textit{Schiller} (test eseguito durante la colposcopia) e \textit{Citology} (osservazione al microscopio di cellule ottenute mediante il Pap test o altre tecniche).
Il processo di addestramento dei modelli è stato eseguito nelle medesime condizioni sperimentali dei test precedenti, ovvero utilizzando un processo di 5-fold \textit{stratified cross-validation} con \textit{oversampling} sul train set.

\section{RISULTATI}\begin{figure}
	\centering
	\includegraphics[width=0.9\linewidth]{images/pca-perf}
	\caption{}
	\label{fig:pca-perf}
\end{figure}


\addtolength{\textheight}{-5cm}

\section{CONCLUSIONI E SVILUPPI FUTURI}
Nel lavoro appena presentato è stato analizzato un dataset derivante dai dati di $858$ pazienti di un ospedale di Caracas che si sono sottoposte ad approfondimenti per la prevenzione del tumore alla cervice uterina.
Dopo aver eseguito analisi esplorative del dataset, che hanno mostrato un forte sbilanciamento dei sample verso la non presenza del tumore, il dataset è stato sottoposto a operazioni di \textit{cleaning} e \textit{preprocessing}, rimuovendo due feature (a causa della scarsa rappresentazione nei record) e effettuando operazioni di imputazioni sui molti \textit{missing values} presenti nel dataset.
A seguito della rimozione degli outlier, sono state eseguite analisi esplorative dei record che presentavano biopsia positiva, analizzando la distribuzione dei fattori di rischio noti in letteratura.
L'analisi effettuata ha in parte confermato le informazioni apprese sulla malattia, evidenziando un range di età compatibile e una possibile relazione tra malattie sessualmente trasmissibili e lo sviluppo di questa forma di cancro.
L'analisi della correlazione tra le feature ha evidenziato valori elevati per molte di esse.\\
Terminate le operazioni di \textit{preprocessing} sul dataset, sono stati utilizzati due differenti algoritmi di apprendimento automatico, decision tree e random forest, per effettuare la classificazione dei record.
Al fine di stabilire se una tecnica di feature reduction potesse portare beneficio al processo di addestramento degli algoritmi, sono state realizzate tre differenti pipeline parallele; la prima utilizza l'intero dataset, la seconda applica una PCA e la terza un filtro che sfrutta la correlazione.
Il numero di componenti della PCA e il threshold del filtro sono stati stabiliti mediante un'indagine delle performance dei modelli a seguito della variazione di questi valori.
Stabiliti gli iperparametri ottimali per le tecniche di riduzione della dimensionalità, gli algoritmi sono stati addestrati in un processo di 5-fold \textit{startified cross-validation}, utilizzando SMOTE come tecnica di \textit{oversampling} sulla porzione di train del dataset.
L'utilizzo di tale tecnica è motivato da un'analisi qualitativa che ha mostrato un non peggioramento delle performance degli algoritmi, con miglioramenti in alcuni casi.
I risultati delle random forest hanno mostrato performance medie leggermente superiori rispetto ai rispettivi alberi, ma i test di significatività non hanno rilevato una differenza statisticamente significativa (fatta eccezione per i modelli con in input il dataset ridotto tramite filtro). Anche un confronto tra modelli con input differenti non ha evidenziato una differenza significativa tra le medie.
Infine, è stato effettuato un tentativo di addestrare gli algoritmi precedenti in assenza delle feature relative agli esami clinici effettuati sulle pazienti. Le performance ottenute hanno mostrato un drastico deterioramento, confermando la necessità di queste analisi al fine di un corretto riconoscimento della patologia.
Concludendo, è possibile affermare che gli obiettivi posti all'inizio di questo lavoro siano stati raggiunti, anche se i test statistici non hanno evidenziato con chiarezza la superiorità di uno tra i vari approcci proposti.
Al fine si superare questo problema, in futuro sarebbe necessario raccogliere dati su una popolazione sensibilmente più ampia, in modo da non essere più limitati dalla scarsa rappresentazione di record che presentano la patologia e rendendo altresì più significative le analisi effettuate e più robusti i modelli proposti.
Lo sviluppo di una pipeline per l'imputazione dei \textit{missing value} che tenga in considerazione i legami e le relazioni tra le varie feature potrebbe garantire un ulteriore margine di miglioramento, assicurando la consistenza di alcuni vincoli logici che andrebbero altrimenti persi con le classiche tecniche di imputazione.
Un ulteriore punto di sviluppo di questo lavoro potrebbe scaturire dal confronto con esperti del dominio medico, grazie ai quali si potrebbe ottenere una stima più accurata del peso/costo dei diversi tipi di predizioni errate dei modelli, ottimizzandoli così secondo i nuovi criteri proposti.

\section*{NOTE TECNICHE}
Si segnala per questioni di compatibilità che la versione di Knime utilizzata per l'esecuzione del workflow relativo al lavoro svolto è la 4.1.0, con integrazione Python per l'esecuzione di snippet grafici e Weka per la creazione di alcuni modelli di predizione.
Al fine del corretto funzionamento del workflow, è necessario inserire il dataset ottenuto da \cite{ML} nella radice del workspace del progetto e creare una cartella denominata images, nella quale verranno salvati i grafici prodotti dal workflow.
La repository del progetto è disponibile all'indirizzo\\\url{https://gitlab.com/mistrello96/cervical_cancer_prediction}


\begin{thebibliography}{99}

\bibitem{paper}
Kelwin Fernandes, Jaime S Cardoso, and Jessica Fernandes.
\textit{Transfer learning with partial observability applied to cervical cancer screening}.
Iberian conference on pattern recognition and image analysis, 243--250, 2017.

\bibitem{veronesi}
Fondazione Umberto Veronesi, Tumore della cervice uterina.
\url{https://www.fondazioneveronesi.it/magazine/tools-della-salute/glossario-delle-malattie/tumore-della-cervice-uterina-2}

\bibitem{ML}
UCI, Machine Learning Repository, Cervical cancer (Risk Factors) Data Set.
\url{https://archive.ics.uci.edu/ml/datasets/Cervical+cancer+%28Risk+Factors%29}

\end{thebibliography}

\end{document}
